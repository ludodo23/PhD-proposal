\documentclass[12pt]{article}
\usepackage[a4paper,margin=2.5cm]{geometry}
\usepackage[english]{babel}
\usepackage[utf8]{inputenc}
\usepackage[T1]{fontenc}
\usepackage{lmodern}
\usepackage{csquotes}
\usepackage[colorlinks=true, linkcolor=black, citecolor=black, urlcolor=blue]{hyperref}
\usepackage[acronym]{glossaries}
\usepackage[most]{tcolorbox}
\usepackage{tabularx}

\makeglossaries

\newacronym{cfd}{CFD}{Computational Fluid Dynamics}
\newacronym{edp}{PDE}{Partial Differential Equation}
\newacronym{gnc}{GNC}{Guidance, Navigation and Control}
\newacronym{pinn}{PINN}{Physics-Informed Neural Network}
\newacronym{pino}{PINO}{Physics-Informed Neural Operator}
\newacronym{piml}{PIML}{Physics-Informed Machine Learning}
\newacronym{ml}{ML}{Machine Learning}
\newacronym{ffnn}{FFNN}{Feedforward Neural Network}
\newacronym{no}{NO}{Neural Operator}
\newacronym{ecss}{ECSS}{European Cooperation for Space Standardization}
\newacronym{smp}{SMP}{Simulation Model Portability}
\newacronym{esa}{ESA}{European Space Agency}
\newacronym{cnes}{CNES}{Centre National d'Études Spatiales}
\newacronym{inria}{INRIA}{Institut National de Recherche en Informatique et en Automatique}
\newacronym{onera}{ONERA}{Office National d'Études et de Recherches Aérospatiales}

\AfterPreamble{\hypersetup{
		pdfauthor={Ludovic Andrieux},
		pdftitle={PhD proposal},
		pdfsubject={Multi-fidelity modeling of propellant sloshing in space launcher tanks through the hybridization of analytical models, CFD, and physics-informed neural networks.}
}}

% Bibliography
\usepackage[
backend=biber,        % default compiler for biblatex
sorting=nyt,          % sort by name, year, title
citestyle=numeric, % citation style author-year
bibstyle=numeric,  % bibliography style alphabetic
]{biblatex}

\addbibresource{ESA.bib} % Your .bib file name

\title{PhD Proposal}
\author{Ludovic Andrieux}
\date{\today}

\begin{document}
	\maketitle
	
	\begin{tcolorbox}[colback=gray!0, colframe=black, sharp corners, boxrule=0.5pt]
		Multi-fidelity modeling of propellant sloshing in space launcher tanks through the hybridization of analytical models, \acrshort{cfd}, and physics-informed neural networks.
	\end{tcolorbox}
	
	\section*{Keywords}
	
	Propellant sloshing, space tanks, \acrshort{cfd}, pendulum models, \acrshort{pinn}, \acrshort{gnc} simulation, \acrshort{ecss}/\acrshort{smp}
	
	\section*{Desired Profile and Skills}
	
	\begin{itemize}
		\item Engineering degree or Master's degree in aerospace or applied data science.
		
		\item Skills in fluid mechanics, numerical simulation (\acrshort{cfd}), reduced-order modeling methods (mass-spring model, pendulum).
		
		\item Interest or experience in physics-informed machine learning (\acrshort{pinn}, \acrshort{piml}).
		
		\item Proficiency in C++ and/or Python, familiarity with \acrshort{ecss}/\acrshort{smp} standards is a plus.
	\end{itemize}
	
	\section*{PhD Project Description}
	
	\gls{pinn}s represent a new paradigm in deep learning, capable of solving both direct and inverse problems for nonlinear \gls{edp}s \cite{raissiPhysicsinformedNeuralNetworks2019}. By integrating underlying physical constraints into the architecture of a \gls{ffnn}, \gls{pinn}s can be trained as surrogate models with little to no labeled data for inferring solutions to \gls{edp}s \cite{cuomoScientificMachineLearning2022}. In the current literature, the implementation of \gls{pinn} is seen both as a complement and a potential alternative to existing numerical techniques, across a wide range of research fields in science and engineering \cite{maoPhysicsinformedNeuralNetworks2020, buosoPersonalisingLeftventricularBiophysical2021, caiPhysicsInformedNeuralNetworks2021}.
	
	\subsection*{Context}
	
	The behavior of propellants in space launcher tanks directly impacts flight stability, the performance of flight control algorithms, and mission success. During propelled phases, the dynamic effects of sloshing masses can be efficiently modeled using equivalent pendulum models (see for example \cite{ibrahimLiquidSloshingDynamics2005a}), but these models become limited when the acceleration regime becomes too weak (and even non-functional in micro-gravity), the sloshing motion too large, or when dominant biphasic behavior occurs.
	
	\subsection*{Problem}
	
	In a numerical simulation context, especially for flight control validation and qualification, it is essential to have models that are both accurate, fast, and adaptable, capable of representing physical phenomena at different scales and under various flight conditions.
	The current approaches struggle to provide a generic, fast, and reliable framework for modeling propellant sloshing during all flight phases. \acrshort{cfd} provides accurate but costly results, reduced models (pendulums, mass-spring) are effective during powered phases with relatively small sloshing angles, while \gls{pinn} offer a promising trade-off between accuracy and cost.
	
	\subsection*{Objectives}
	
	This thesis aims to develop a generic framework for modeling propellant behavior in tanks, integrating multiple fidelity levels, and usable in standard industrial environments (fully numerical simulation or with physical elements), with a focus on cases where reduced models are unavailable (micro-gravity and launcher dynamics during powered phases outside pendulum model applicability).
	
	The main objectives are:
	\begin{enumerate}
		\item Develop a modular C++ software architecture:
		\begin{itemize}
			\item Pendulum and mass-spring models,
			\item \acrshort{cfd} models (based on the \acrshort{cfd} codes used for the construction of the dataset),
			\item Integration of \gls{pinn}/\acrshort{piml} models,
			\item Interoperability with \acrshort{ecss}/\acrshort{smp} standards.
		\end{itemize}
		\item Explore, compare, and synthesize sloshing modeling methods:
		\begin{itemize}
			\item Analytical approach (small angles, linear, pendulum),
			\item Numerical \acrshort{cfd} simulation (compressible gas and liquid with free surface, evaporation and thermal transport, extraction of wall forces),
			\item \acrfull{pinn}, \acrfull{no} and \acrfull{pino}.
		\end{itemize}
		\item Propose a model selection strategy based on:
		\begin{itemize}
			\item Required fidelity level,
			\item Simulation constraints,
			\item Flight phase and environment.
		\end{itemize}
	\end{enumerate}
	
	\section*{Methodological Approach}
	
	The thesis will rely on a structured approach divided into three main components:
	\begin{enumerate}
		\item Physical and numerical modeling of the fluid mechanics problem, with the aim of defining test cases (constant accelerations, transients, micro-gravity) and implementing them to, on one hand, characterize analytical models (pendulum and/or mass-spring), and on the other hand, generate the datasets for neural network models.
		
		An incremental approach may be envisioned in order to gradually refine the model with a growing complexity of phenomena to implement:
		\begin{itemize}
			\item incompressible gas-liquid system with a free surface,
			\item gas compressibility and source terms,
			\item thermal transport and phase change.
		\end{itemize}
		
		\item Development of full (\acrshort{cfd}), reduced (pendulum), surrogate (\acrshort{pinn} or \acrshort{piml}), and hybrid models according to \acrshort{ecss}/\acrshort{smp} standards for integration into an industrial environment that ensures good portability across various simulation environments.
		
		\item Performance evaluation on the trade-off between accuracy and speed through integration of models into a closed-loop digital simulator and development of recommendations for model selection based on usage.
	\end{enumerate}
	
	\section*{Proposed Approach}
	
	\begin{table}[h]
		\centering
		\begin{tabularx}{\textwidth}{l X}
			
			\hline
			
			Period & Step \\
			
			\hline
			
			Semester 1 & Literature review, construction of test cases, mathematical formulation of the problem \\
			Semester 2 & \acrshort{cfd} simulations, data extraction, initial \acrshort{pinn} prototypes \\
			Semester 3 & Development of the modular library, integration of \acrshort{pinn} and pendulum models \\
			Semester 4 & Evaluation, benchmarks, model selection strategy \\
			Semester 5 & Integration into simulators \\
			Semester 6 & Finalization, defense, software submission if applicable \\
			
			\hline
			
		\end{tabularx}
		\caption{Project timeline}
	\end{table}	
	
	\section*{Scientific and Industrial Contributions}
	
	Accurate and fast evaluation tools for the modeling of propellants in launch vehicle tanks are expected.
	
	In addition, this synthetic approach is expected to provide deeper insights into the limitations and complementarities of analytical models, \acrshort{cfd}, and \acrlong{ml}.
	
	Specifically, a direct contribution is expected regarding the modeling of the physical phenomenon:
	\begin{itemize}
		\item A reduced model for sloshing in micro-gravity: where the state of the art does not provide anything usable in an environment coupled with \acrshort{gnc} algorithms, unlike phases under acceleration.
		\item Initialization of pendulum models after an orbital phase: where the current state of knowledge forces us to randomly generate the initial state. The goal is to reduce such conservatism.
		\item Implementation of models during the flip maneuver of reusable launchers: where the dynamics of the launcher may impact sloshing behavior, moving beyond the applicability domain of pendulum models.
	\end{itemize}
	A direct contribution to the development of a launcher project, with the ability to simulate complete cases typically simulated via \acrshort{cfd} coupling later in the project.
	
	Finally, through integration into an industrial environment, we expect significant operational gains.
	
	\nocite{*}
	
	\printbibliography[title=Indicative Bibliography]
	
	\section*{Envisioned Collaborations}
	
	\acrshort{esa}, \acrshort{cnes}, \acrshort{inria}, \acrshort{onera}
	
	\section*{Host Laboratory}
	
	TBD
	
	\printglossaries
	
\end{document}
