\documentclass{beamer}
\usepackage[utf8]{inputenc}
\usepackage[french]{babel}
\usetheme{Madrid}
\title[Proposition de thèse]{Modélisation multi-fidélité des ballottements d’ergols}
\subtitle{Hybridation de modèles analytiques, CFD et réseaux de neurones informés par la physique}
\author{Ludovic Andrieux}
\date{24 juin 2025}

\begin{document}
	
	\frame{\titlepage}
	
	\begin{frame}{Objectifs de la thèse}
		\begin{itemize}
			\item Développer un cadre générique pour modéliser le comportement des ergols dans les réservoirs spatiaux.
			\item Intégrer plusieurs niveaux de fidélité :
			\begin{itemize}
				\item Modèles analytiques (pendulaires, masse-ressort),
				\item Simulation CFD (gaz/liquide, interface libre, transfert thermique),
				\item Réseaux de neurones informés par la physique (PINN, NO, PINO).
			\end{itemize}
			\item Assurer la portabilité dans des environnements industriels (interopérabilité ECSS/SMP).
			\item Évaluer le compromis précision/rapidité pour des cas réalistes (simulateur numérique).
		\end{itemize}
	\end{frame}
	
	\begin{frame}{Approche et retombées}
		\textbf{Approche structurée en 3 volets :}
		\begin{enumerate}
			\item \textbf{Modélisation physique} : cas tests et données pour l’apprentissage.
			\item \textbf{Développement multi-modèles} : CFD, réduits, PINN, hybrides.
			\item \textbf{Évaluation des performances} : benchmarks, stratégie de sélection.
		\end{enumerate}
		\vspace{0.5cm}
		\textbf{Apports attendus :}
		\begin{itemize}
			\item Meilleure modélisation en micro-gravité et phases de retournement.
			\item Réduction des conservatismes dans l’initialisation des modèles.
			\item Gains opérationnels grâce à l’intégration dans des simulateurs industriels.
		\end{itemize}
	\end{frame}
	
\end{document}
