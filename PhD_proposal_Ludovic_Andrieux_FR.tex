\documentclass[12pt]{article}
\usepackage[a4paper,margin=2.5cm]{geometry}
\usepackage[french]{babel}
\usepackage[utf8]{inputenc}
\usepackage[T1]{fontenc}
\usepackage{lmodern}
\usepackage{csquotes}
\usepackage[colorlinks=true, linkcolor=black, citecolor=black, urlcolor=blue]{hyperref}
\usepackage[acronym]{glossaries}
\usepackage[most]{tcolorbox}
\usepackage{tabularx}

\makeglossaries

\newacronym{cfd}{CFD}{Computational Fluid Dynamics}
\newacronym{edp}{EDP}{Equation aux Dérivées Partielles}
\newacronym{edps}{EDPs}{Equations aux Dérivées Partielles}
\newacronym{gnc}{GNC}{Guidance, Navigation and Control}
\newacronym{pinn}{PINN}{Physics-Informed Neural Network}
\newacronym{bpinn}{B-PINN}{Bayesian Physics-Informed Neural Network}
\newacronym{pino}{PINO}{Physics-Informed Neural Operator}
\newacronym{piml}{PIML}{Physics-Informed Machine Learning}
\newacronym{sciml}{SciML}{Scientific Machine Learning}
\newacronym{ml}{ML}{Machine Learning}
\newacronym{ffnn}{FFNN}{Feedforward Neural Network}
\newacronym{no}{NO}{Neural Operator}
\newacronym{ecss}{ECSS}{European Cooperation for Space Standardization}
\newacronym{smp}{SMP}{Simulation Model Portability}
\newacronym{esa}{ESA}{European Space Agency}
\newacronym{cnes}{CNES}{Centre National d'Études Spatiales}
\newacronym{inria}{INRIA}{Institut National de Recherche en Informatique et en Automatique}
\newacronym{onera}{ONERA}{Office National d'Études et de Recherches Aérospatiales}
\newacronym{mpc}{MPC}{Model Predictive Control}
\newacronym{trl}{TRL}{Technology Readiness Level}
\newacronym{uq}{UQ}{Uncertainty Quantification}
\newacronym{vv}{V\&V}{Verification and Validation}



\AfterPreamble{\hypersetup{
		pdfauthor={Ludovic Andrieux},
		pdftitle={Proposition de sujet de thèse},
		pdfsubject={Modélisation des ballottements d’ergols dans les réservoirs de véhicules spatiaux en condition de micro-gravité ou sous faible facteur de charge par apprentissage automatique informé par la physique.
		}
}}

% Bibliographie
\usepackage[
backend=biber,        % compilateur par défaut pour biblatex
sorting=nyt,          % trier par nom, année, titre
citestyle=numeric, % style de citation auteur-année
bibstyle=numeric,  % style de bibliographie alphabétique
]{biblatex}

\addbibresource{ESA.bib} % Le nom de ton fichier .bib
\addbibresource{UQ.bib} % Your .bib file name

\title{Proposition de sujet de thèse}
\author{Ludovic Andrieux}
\date{\today}



\begin{document}
	\maketitle
	
	\begin{tcolorbox}[colback=gray!0, colframe=black, sharp corners, boxrule=0.5pt]
	Modélisation des ballottements d’ergols dans les réservoirs de véhicule spatiaux en conditions de micro-gravité ou sous faible facteur de charge par apprentissage automatique informé par la physique.
	\end{tcolorbox}
	
	\section*{Mots-clés}
	
	Ballottements d'ergols, réservoirs spatiaux, \acrshort{cfd}, modèles pendulaires, \acrshort{pinn}, simulation \acrshort{gnc}, \acrshort{ecss}/\acrshort{smp}
	
	\section*{Profil et compétences recherchées}
	
	\begin{itemize}
		\item Diplôme d’ingénieur ou Master 2 en aérospatial ou data science appliquée.
		
		\item Compétences en mécanique des fluides, simulation numérique (\acrshort{cfd}), méthodes de modélisation réduite (modèle masse-ressort, pendule).
		
		\item Intérêt ou expérience en machine learning physique (\acrshort{pinn}, \acrshort{piml} and \acrshort{sciml}).
		
		\item Maîtrise du langage C++ et/ou Python, familiarité avec les standards \acrshort{ecss}/\acrshort{smp} appréciée.
	\end{itemize}
	
	
	\section*{Présentation du projet doctoral}
	
	Les réseaux de neurones informés par la physique, ou \glspl{pinn} constituent un nouveau paradigme de l’apprentissage profond, capable de résoudre à la fois des problèmes directs et inverses pour des \glspl{edp} non linéaires \cite{raissiPhysicsinformedNeuralNetworks2019}. En intégrant des contraintes physiques sous-jacentes dans l’architecture d’un réseau de neurones à propagation avant, ou \gls{ffnn}, les \glspl{pinn} peuvent être entraînés comme modèle de substitution avec peu ou pas de données étiquetées pour l’inférence de solutions d’\gls{edp} \cite{cuomoScientificMachineLearning2022}. Dans la littérature actuelle, l’implémentation des \glspl{pinn} est envisagée à la fois comme complément et comme alternative potentielle aux techniques numériques existantes, dans un large éventail de domaines de recherche en sciences et en ingénierie \cite{maoPhysicsinformedNeuralNetworks2020,buosoPersonalisingLeftventricularBiophysical2021, caiPhysicsInformedNeuralNetworks2021}.
	En outre, certains designs prennent nativement en compte la gestion des incertitudes, ou \gls{uq} \cite{yangBPINNsBayesianPhysicsInformed2021,zhangQuantifyingTotalUncertainty2018} ce qui fait des \glspl{pinn} un outil intéressant en tant que modèle de substitution en ingénierie, où la robustesse des solutions fournies par un code de calcul constitue un sujet important.
	
	\subsection*{Contexte}
	
	
	Le comportement des ergols dans les réservoirs de lanceurs spatiaux a un impact direct sur la stabilité du vol, la performance des algorithmes de contrôle de vol, et la réussite de la mission. Lors des phases propulsées, les effets dynamiques des masses ballottantes peuvent être modélisés efficacement à l'aide de pendules équivalents (voir par exemple \cite{ibrahimLiquidSloshingDynamics2005a}), mais ces modèles deviennent limités dès que le régime d’accélération devient trop faible (et même inopérant en micro-gravité), le mouvement des masses ballottantes trop important, ou encore le comportement diphasique dominant.
	
	\subsection*{Problématique}
	
	Dans un contexte de simulation numérique, notamment dans un cadre de validation et de qualification du contrôle de vol, il est essentiel de disposer de modèles à la fois précis, rapides et adaptables, capables de représenter les phénomènes physiques à différentes échelles et dans différents contextes de vol.
	Les approches actuelles peinent à fournir un cadre générique, rapide et fiable pour modéliser les ballottements d’ergols dans toutes les phases de vol. La \acrshort{cfd} offre des résultats précis mais coûteux, les modèles réduits (typiquement les modèles de pendules) sont efficaces lors des phases propulsées mais sous des conditions d'angles de ballottements relativement faibles (excluant les phases de retournement de lanceur réutilisable), tandis que les réseaux de neurones informés par la physique (\gls{pinn}) offrent un compromis prometteur entre précision et coût.
	
	\subsection*{Objectifs}
	
	Cette thèse vise à développer des modèles réduits pour les ballottements d'ergols dans les réservoirs de véhicules spatiaux en micro-gravité et sous faible accélération à l'aide de \acrfull{piml} puisqu'aucun standard n'existe.
	
	Les objectifs principaux sont :
	\begin{enumerate}
		\item Développer une architecture logicielle C++ modulaire :
		\begin{itemize}
			\item Modèles \acrshort{cfd} (pour la constitution du jeu de données et les simulations boucle fermée end-to-end),
			\item Intégration de modèles \gls{pinn}/\acrshort{piml},
			\item Interopérabilité avec les standards \acrshort{ecss}/\acrshort{smp} et les modèles pendulaires et masse-ressort existants.
		\end{itemize}
		\item Explorer, comparer et synthétiser les méthodes de modélisation des ballottements :
		\begin{itemize}
			\item Approche analytique (petits angles, linéaires, pendules),
			\item Simulation numérique \acrshort{cfd} (gaz compressible et liquide avec interface libre, évaporation et transfert thermique, extraction d'efforts sur les parois),
			\item Réseaux de neurones informés par la physique (\gls{pinn}) et opérateurs neuronaux (\acrshort{no} et \acrshort{pino}).
			\item Incertitude et robustesse des modèles réduits au travers du MC-dropout, méthode d'ensembles, \acrshort{bpinn}, ou décomposition de domaine.
		\end{itemize}
		\item Proposer une stratégie de sélection de modèle selon :
		\begin{itemize}
			\item Le niveau de fidélité requis,
			\item Les contraintes de simulation,
			\item La phase de vol et l'environnement.
		\end{itemize}
	\end{enumerate}
	
	
	\section*{Approche méthodologique}
	
	La thèse s’appuiera sur une approche structurée en trois volets :
	\begin{enumerate}
		\item Les modélisation physiques et numériques du problème de mécanique des fluides, dans le but de définir des cas tests (accélérations à faible facteur de charge et micro-gravité) et les mettre en œuvre pour constituer le jeu de données pour les modèles neuronaux.
		
		Une approche incrémentale est envisagée pour enrichir la modélisation au fur et à mesure d'une monté en complexité des phénomènes à prendre en compte lors de l'implémentation :
		\begin{itemize}
			\item système gaz-liquide incompressible et interface libre,
			\item compressibilité du gaz et représentativité des sources (représsurisation et écoulement sortant),
			\item transfert thermique et changement de phase.
		\end{itemize}
		
		\item Le développement de modèles complets (\acrshort{cfd}) et réduits (\gls{pinn} ou \acrshort{piml}) selon les standards \acrshort{ecss}/\acrshort{smp} pour une intégration dans un environnement industriel qui assurent une bonne portabilité dans différents environnements de simulation.
		
		\item L'évaluation des performances sur le compromis précision (avec quantification des incertitudes) / rapidité à l'aide d'intégration des modèles dans un simulateur numérique boucle fermée et établissement de recommandations pour le choix de modèle selon l'usage.	
		
	\end{enumerate}
	
	\section*{Démarche envisagée}
	
	\begin{table}[h]
		\centering
		\begin{tabularx}{\textwidth}{l X}
			
			\hline
			
			Période & Étape \\
			
			\hline
			
			Semestre 1 & Revue bibliographique, construction des cas tests et formulation mathématique du problème \\
			Semestre 2 & Simulations \acrshort{cfd}, extraction de données, premiers prototypes \acrshort{pinn} \\
			Semestre 3 & Développement de la librairie modulaire, intégration de modèles \acrshort{pinn} et \acrshort{cfd} \\
			Semestre 4 & Évaluation, étude comparative, stratégie de sélection de modèle \\
			Semestre 5 & Intégration dans simulateur de référence tout numérique à sélectionner, étude de performance sur simulation complète \\
			Semestre 6 & Finalisation, soutenance, dépôt logiciel si applicable \\
			
			\hline
			
		\end{tabularx}
		\caption{Planning prévisionnel des étapes du projet}
	\end{table}	
	
	\section*{Apports scientifiques et industriels}
	
	Des outils d'évaluation à la fois rapides et réalistes de modélisation d'ergols dans les réservoirs de véhicule spatiaux sont attendus.
	
	Cette approche synthétique a pour objectif de mieux cerner les domaines de validité, les interactions et les complémentarités entre modèles analytiques, simulations \acrshort{cfd} et techniques de \acrlong{ml}
	
	Plus spécifiquement, un apport direct est attendu en ce qui concerne la modélisation du phénomène physique. Typiquement :
	\begin{itemize}
		\item Une modélisation réduite des ballottements en micro-gravité : où l'état de l'art ne fournit pas de modèle réduit qui soit utilisable dans un environnement couplé avec des algorithmes \acrshort{gnc}, contrairement aux phases sous accélération.
		\item Des modèles de ballottements embarquable: permettre l'intégration dans un code de vol de modèles précis pour le pilotage et la simulation en temps réel.
		\item L'initialisation des modèles pendulaires après une phase orbitale : où l'état actuel des connaissances nous pousse à tirer aléatoirement cet état initial. On cherche alors à réduire ces conservatismes.
		\item La mise en œuvre de modèles lors des phases de retournement des lanceurs réutilisables : où la dynamique du lanceur peut avoir un impact sur celle des ballottements qui fait sortir du domaine d'applicabilité des modèles pendulaires.
	\end{itemize}
	Un apport direct sur le déroulement d'un projet de développement de lanceur est aussi attendu, avec la capacité de simuler à l'aide de modèles \acrshort{ml} des cas complets qui sont généralement simulés par couplage \acrshort{cfd} tard dans le déroulement du projet.
	
	Enfin, au travers de l'intégration dans un environnement industriel, on estime un gain opérationnel conséquent.
	
	\subsection*{Innovation}
	
	Bien que les \glspl{pinn} aient été appliqués avec succès aux écoulements à surface libre terrestres et à des bancs d’essai fluides diphasiques génériques, aucun travail publié ne traite du ballottement des ergols dans les réservoirs en micro-gravité ou lors des manœuvres en faible gravité des lanceurs, en incluant la physique cryogénique diphasique compressible ainsi que les effets thermiques/de changement de phase, avec des modèles de substitution prêts à être intégrés aux contrôleurs et conformes aux normes \acrshort{ecss}/\acrshort{smp}.
	Les travaux existants liés à l’aérospatial sont soit non informés par la physique (utilisant uniquement l’apprentissage profond, moins explicables, et sans la robustesse requise par les processus de \acrshort{vv}), soit limités à des problèmes simplifiés (excluant la cryogénie et les conditions spatiales telles que la microgravité).
	
	
	
	Cette thèse présente une combinaison unique de technique et de cas d’utilisation :
	
	\begin{itemize}
		\item Technique : Fait progresser l’état de l’art \acrshort{pinn}/\acrshort{pino} pour la capture d’\acrshort{edps} non linéaires multiphasiques décrivant la physique des ergols embarqués (interactions gaz–liquide compressibles, dynamique de surface libre, transport thermique et changement de phase, géométrie des réservoirs et conditions aux limites), avec des propriétés de différentiabilité exploitables pour le \gls{mpc}.
		
		\item Cas d’utilisation : Réservoirs de véhicules spatiaux pendant les phases de micro-gravité, de renversement ou de transition, pour lesquels aucun modèle analytique ou réduit n’est actuellement opérationnel.
	\end{itemize}
	
	
	\subsection*{Degré de maturité}
	
	
	État de l’art : Les \acrshort{pinn}s sont à un \acrshort{trl} de 2–3 en mécanique des fluides ; validés pour des \acrshort{pinn}s plus simples, pour des écoulements diphasiques non cryogéniques, et dans des domaines non liés à l’aérospatial.
	\\
	
	Lacune : Aucune application aux réservoirs aérospatiaux avec thermodynamique cryogénique, aucune intégration dans les environnements de simulation industrielle, aucune insertion en boucle de contrôle.
	\\
	
	Adéquation avec une thèse :
	
	\begin{itemize}
		\item Les modèles \acrshort{cfd} et analytiques pour la génération de jeux de données sont disponibles auprès de l’\acrshort{esa}/\acrshort{cnes}/\acrshort{onera}.
		
		\item Les chaînes d’outils \acrshort{pinn}/\acrshort{pino} (PyTorch/TensorFlow, JAX, DeepXDE) sont suffisamment matures pour un prototypage immédiat.
		
		\item Les spécifications d’interface \acrshort{ecss}/\acrshort{smp} sont publiques et peuvent être mises en œuvre en parallèle du développement des modèles.
		
	\end{itemize}	
	
	Un arc de recherche de 3–4 ans permet d’approfondir la complexité physique (changement de phase, ébullition), d’améliorer la stabilité de l’apprentissage des modèles de substitution et d’effectuer une validation en boucle fermée.
	
	\textbf{TRL cible après la thèse : 4–5} — modèle \acrshort{pinn} validé dans une simulation de bout en bout avec logiciel représentatif en boucle fermée, prêt pour une extension vers des bancs d’essai représentatifs du vol.
	
	
	\nocite{*}
	
	\printbibliography[title=Bibliographie indicative]
	
	\section*{Collaborations envisagées}
	
	\acrshort{esa}, \acrshort{cnes}, \acrshort{inria}, \acrshort{onera}
	
	\section*{Laboratoire d'accueil}
	
	TBD
	
	\printglossaries
	
\end{document}